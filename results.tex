%\documentclass{article}
%\usepackage[utf8]{inputenc}
%\usepackage{amsmath}
%\usepackage{natbib}
%\usepackage{graphicx}
%\usepackage{astrojournals} % Necesario para nombres de revistas en luis-ref.bib
%\usepackage[spanish, es-minimal]{babel}
%\usepackage{longtable}
%\usepackage{geometry}
%\usepackage{multirow, array}
%\newlength\figwidth
%\setlength\figwidth{0.48\textwidth}
%\bibliographystyle{apj}


%\title{Catalog of stationary bowshock arcs in the Orion Nebula}

%\author{
  %Alumno: Luis Angel Gutiérrez Soto\\
  %Tutor: Dr. William Henney
%}
%\begin{document}
%\maketitle


%\chapter{Resultados}
\label{chap:results}

\section{Resultados empíricos}
\label{sec:results-1}

\subsection{Catálogo: Imágenes de los arcos de proa estacionarios}
\label{sec:images}


En este trabajo hemos detectado 73 objetos, los cuales hemos clasificado como los arcos de proa estacionarios que conforman nuestro catálogo. Entonces la  figura \ref{fig:images} es el conjunto de todas las imágenes de los objetos LL y de los proplyds con sus respectivos arcos de emisión en la Nebulosa de Orión. En esta imágenes se ilustra la diversidad morfológica de los choques LL. En este sentido mencionaremos a continuación  las características morfológicas más percepcibles en algunos objetos de nuestro catálogo (estos objetos los consideremos interesantes, debido a sus particulares estructuras):\\

\begin{itemize}
\item LL1 es el prototipo de los objetos LL  (también llamado LL~Ori como se mensionó en el capítulo~\ref{chap:introduction}).
\item Algunos de nuestros objetos presentan doble cáscaras, es el caso de los objetos: LL3, LL5, 4578-251, 072-134, w266-558 y 362-3137 en este último no estamos seguros de si trata de una doble zona chocada.  
\item Los objetos LL2, LL6, 203-3039 y probablemente LL4 presentan un jet perpendicular a la dirección hacia donde están orientados los choques.
\item Algunos choques de proa tienen jets paralelos a su eje, es el caso de los objetos 4468-605 y w044-527.
\item El choque w000-400 tiene las alas de su arco muy extendidas.
\item Las cáscaras del choque w012-407 es débil, mientras que la de los choques; 189-329 y 1039-3057 resultan ser mucho más débiles.
\item Los objetos w014-414 y 049-143 muestran cáscaras chocadas muy anchas.
\item En el choque 101-233 se observa que su cáscara está compuesta por grumos.
\item Al arco del objeto 109-2146, al parecer se le superpone un objeto HH.
\item Se observa que la cáscara de 119-3155 está interrumpida, es decir la forma de los arcos no es continua.
\item Los objetos 168-328, 166-316 y 158-323 están cerca de otros proplyds. Se altera la forma de los arcos debido a la interacción de los vientos de dos proplyds.  
\item El proplyd 180-331 muestra un choque de proa altamente asimétrico, al igual que el arco del objeto w0044-527 con la pequeña  diferencia de que éste último no es tan asimétrico.
\item El objeto 308-3036 tiene un choque interno con forma circular.
\item Los arcos del proplyd w069-601 describen una parábola perfecta.
\item Los arcos del objeto 116-3101 son muy cerrados, mientras que los arcos del objeto 102-157 son muy abietros.

\end{itemize}


\begin{figure*}[htp]
\centering
\small\raggedright
\renewcommand{\arraystretch}{1.5}
\begin{tabular}{|l|l|}
\hline
   \includegraphics[width=0.47\linewidth,  trim=60 50 100 50, clip]{j8oc01010_wcs/LL1-Bally_01-images.pdf} 
   &\includegraphics[width=0.47\linewidth,  trim=60 50 100 50, clip]{j8oc01010_wcs/065-502-Bally_01-images.pdf}\\ \hline
   \includegraphics[width=0.47\linewidth,  trim=60 50 100 50, clip]{j8oc01010_wcs/074-229-Bally_01-images.pdf}
   &\includegraphics[width=0.47\linewidth,  trim=60 50 100 50, clip]{j8oc01010_wcs/083-435-Bally_01-images.pdf}\\ \hline
   \includegraphics[width=0.47\linewidth,  trim=60 50 100 50, clip]{j8oc01010_wcs/101-233-Bally_01-images.pdf}
   &\includegraphics[width=0.47\linewidth,  trim=60 50 100 50, clip]{j8oc01010_wcs/102-157-Bally_01-images.pdf}\\ \hline
    \includegraphics[width=0.47\linewidth, trim=60 50 100 50, clip]{j8oc01010_wcs/106-245-Bally_01-images.pdf}
   &\includegraphics[width=0.47\linewidth, trim=60 50 100 50, clip]{j8oc01010_wcs/109-246-Bally_01-images.pdf}\\  \hline
   \includegraphics[width=0.47\linewidth, trim=60 50 100 50, clip]{j8oc01010_wcs/117-421-Bally_01-images.pdf}
   &\includegraphics[width=0.47\linewidth, trim=60 50 100 50, clip]{j8oc01010_wcs/121-434-Bally_01-images.pdf}\\ \hline
   \includegraphics[width=0.47\linewidth, trim=60 50 100 50, clip]{j8oc01010_wcs/154-225-Bally_01-images.pdf}
   &\includegraphics[width=0.47\linewidth, trim=60 50 100 50, clip]{j8oc01010_wcs/154-240-Bally_01-images.pdf}\\ \hline
\end{tabular}
\end{figure*}

\begin{figure*}[htp]
\centering
\begin{tabular}{|l|l|}
\hline
   \includegraphics[width=0.47\linewidth,  trim=60 50 100 50, clip]{j8oc01010_wcs/158-323-Bally_01-images.pdf}
   &\includegraphics[width=0.47\linewidth,  trim=60 50 100 50, clip]{j8oc01010_wcs/159-221-Bally_01-images.pdf}\\ \hline
   \includegraphics[width=0.47\linewidth,  trim=60 50 100 50, clip]{j8oc01010_wcs/161-324-Bally_01-images.pdf}
   &\includegraphics[width=0.47\linewidth,  trim=60 50 100 50, clip]{j8oc01010_wcs/163-222-Bally_01-images.pdf}\\ \hline
   \includegraphics[width=0.47\linewidth,  trim=60 50 100 50, clip]{j8oc01010_wcs/163-317-Bally_01-images.pdf}
   &\includegraphics[width=0.47\linewidth,  trim=60 50 100 50, clip]{j8oc01010_wcs/165-235-Bally_01-images.pdf}\\ \hline
    \includegraphics[width=0.47\linewidth,  trim=60 50 100 50, clip]{j8oc01010_wcs/166-316-Bally_01-images.pdf}
   &\includegraphics[width=0.47\linewidth,  trim=60 50 100 50, clip]{j8oc01010_wcs/167-317-Bally_01-images.pdf}\\ \hline
   \includegraphics[width=0.47\linewidth,  trim=60 50 100 50, clip]{j8oc01010_wcs/168-326-Bally_01-images.pdf}
   &\includegraphics[width=0.47\linewidth,  trim=60 50 100 50, clip]{j8oc01010_wcs/168-326N-Bally_01-images.pdf}\\ \hline
   \includegraphics[width=0.47\linewidth,  trim=60 50 100 50, clip]{j8oc01010_wcs/168-328-Bally_01-images.pdf}
 &\includegraphics[width=0.47\linewidth,  trim=60 50 100 50, clip]{j8oc01010_wcs/169-338-Bally_01-images.pdf}\\ \hline
   
 \end{tabular}
\end{figure*}

\begin{figure*}[htp]
\centering
\begin{tabular}{|l|l|}
\hline
    \includegraphics[width=0.47\linewidth,  trim=60 50 100 50, clip]{j8oc01010_wcs/170-249-Bally_01-images.pdf}
   &\includegraphics[width=0.47\linewidth,  trim=60 50 100 50, clip]{j8oc01010_wcs/173-236-Bally_01-images.pdf}\\ \hline
   \includegraphics[width=0.47\linewidth,  trim=60 50 100 50, clip]{j8oc01010_wcs/173-342-Bally_01-images.pdf}
   &\includegraphics[width=0.47\linewidth,  trim=60 50 100 50, clip]{j8oc01010_wcs/175-321-Bally_01-images.pdf}\\ \hline
   \includegraphics[width=0.47\linewidth,  trim=60 50 100 50, clip]{j8oc01010_wcs/177-341-Bally_01-images.pdf}
   &\includegraphics[width=0.47\linewidth,  trim=60 50 100 50, clip]{j8oc01010_wcs/178-258-Bally_01-images.pdf}\\ \hline
   \includegraphics[width=0.47\linewidth,  trim=60 50 100 50, clip]{j8oc01010_wcs/180-331-Bally_01-images.pdf}
   &\includegraphics[width=0.47\linewidth,  trim=60 50 100 50, clip]{j8oc01010_wcs/189-329-Bally_01-images.pdf}\\ \hline
   \includegraphics[width=0.47\linewidth,  trim=60 50 100 50, clip]{j8oc01010_wcs/w012-407-Bally_01-images.pdf}
   &\includegraphics[width=0.47\linewidth,  trim=60 50 100 50, clip]{j8oc01010_wcs/w014-414-Bally_01-images.pdf}\\ \hline
   \includegraphics[width=0.47\linewidth,  trim=60 50 100 50, clip]{j8oc01010_wcs/w030-524-Bally_01-images.pdf}
    &\includegraphics[width=0.47\linewidth,  trim=60 50 100 50, clip]{j8oc01010_wcs/w044-527-Bally_01-images.pdf}\\ \hline
  
 \end{tabular}
\end{figure*}


\begin{figure*}[htp]
\centering
\begin{tabular}{|l|l|}
  \hline
   \includegraphics[width=0.47\linewidth,  trim=60 50 100 50, clip]{j8oc01010_wcs/w069-601-Bally_01-images.pdf}
   &\includegraphics[width=0.47\linewidth,  trim=60 50 100 50, clip]{j8oc01010_wcs/w073-227-Bally_01-images.pdf}\\ \hline
    \includegraphics[width=0.47\linewidth,  trim=60 50 100 50, clip]{j8oc01010_wcs/142-301-Bally_01-images.pdf}
    &\includegraphics[width=0.47\linewidth,  trim=60 50 100 50, clip]{j8oc02010_wcs/131-046-Bally_02-images.pdf}\\ \hline
    \includegraphics[width=0.47\linewidth,  trim=60 50 100 50, clip]{j8oc02010_wcs/132-053-Bally_02-images.pdf}
   &\includegraphics[width=0.47\linewidth,  trim=60 50 100 50, clip]{j8oc02010_wcs/206-043-Bally_02-images.pdf}\\  \hline
    %\includegraphics[width=0.47\linewidth,  trim=60 50 100 50, clip]{j8oc06010_wcs/205-230-Bally_06-images.pdf}
    &\includegraphics[width=0.47\linewidth,  trim=60 50 100 50, clip]{j8oc06010_wcs/212-400-Bally_06-images.pdf}\\ \hline
    \includegraphics[width=0.47\linewidth,  trim=60 50 100 50, clip]{j8oc06010_wcs/w266-558-Bally_06-images.pdf}
    &\includegraphics[width=0.47\linewidth,  trim=60 50 100 50, clip]{j8oc07010_wcs/261-3018-Bally_07-images.pdf}\\ \hline
    \includegraphics[width=0.47\linewidth,  trim=60 50 100 50, clip]{j8oc07010_wcs/305-811-Bally_07-images.pdf}
    &\includegraphics[width=0.47\linewidth,  trim=60 50 100 50, clip]{j8oc07010_wcs/LL5-Bally_07-images.pdf}\\ \hline
\end{tabular}
\end{figure*}

 
\begin{figure*}[htp]
\centering
\begin{tabular}{|l|l|}
\hline
    \includegraphics[width=0.47\linewidth,  trim=60 50 100 50, clip]{j8oc08010_wcs/308-3036-Bally_08-images.pdf}
    &\includegraphics[width=0.47\linewidth,  trim=60 50 100 50, clip]{j8oc08010_wcs/344-3020-Bally_08-images.pdf}\\ \hline
    \includegraphics[width=0.47\linewidth,  trim=60 50 100 50, clip]{j8oc08010_wcs/362-3137-Bally_08-images.pdf}
    &\includegraphics[width=0.47\linewidth,  trim=60 50 100 50, clip]{j8oc08010_wcs/LL6-Bally_08-images.pdf}\\ \hline
    \includegraphics[width=0.47\linewidth,  trim=60 50 100 50, clip]{j8oc09010_wcs/049-143-Bally_09-images.pdf}
    &\includegraphics[width=0.47\linewidth,  trim=60 50 100 50, clip]{j8oc09010_wcs/051-024-Bally_09-images.pdf}\\ \hline
    \includegraphics[width=0.47\linewidth,  trim=60 50 100 50, clip]{j8oc09010_wcs/072-134-Bally_09-images.pdf}
    &\includegraphics[width=0.47\linewidth,  trim=60 50 100 50, clip]{j8oc09010_wcs/4578-251-Bally_09-images.pdf}\\ \hline
    \includegraphics[width=0.47\linewidth,  trim=60 50 100 50, clip]{j8oc09010_wcs/w000-400-Bally_09-images.pdf}
    &\includegraphics[width=0.47\linewidth,  trim=60 50 100 50, clip]{j8oc14010_wcs/066-3251-Bally_14-images.pdf}\\ \hline
    \includegraphics[width=0.47\linewidth,  trim=60 50 100 50, clip]{j8oc14010_wcs/116-3101-Bally_14-images.pdf}
   &\includegraphics[width=0.47\linewidth,  trim=60 50 100 50, clip]{j8oc14010_wcs/119-3155-Bally_14-images.pdf}\\ \hline
    
 \end{tabular}
\end{figure*}

\begin{figure}[htp]
\centering
\begin{tabular}{|l|l|}
\hline
     \includegraphics[width=0.47\linewidth,  trim=60 50 100 50, clip]{j8oc14010_wcs/136-3057-Bally_14-images.pdf}
    &\includegraphics[width=0.47\linewidth,  trim=60 50 100 50, clip]{j8oc14010_wcs/138-3024-Bally_14-images.pdf}\\ \hline
    \includegraphics[width=0.47\linewidth,  trim=60 50 100 50, clip]{j8oc14010_wcs/203-3039-Bally_14-images.pdf}
    &\includegraphics[width=0.47\linewidth,  trim=60 50 100 50, clip]{j8oc16010_wcs/022-635-Bally_16-images.pdf}\\ \hline
    \includegraphics[width=0.47\linewidth,  trim=60 50 100 50, clip]{j8oc16010_wcs/041-637-Bally_16-images.pdf} 
    &\includegraphics[width=0.47\linewidth,  trim=60 50 100 50, clip]{j8oc16010_wcs/042-628-Bally_16-images.pdf}\\ \hline
     \includegraphics[width=0.47\linewidth,  trim=60 50 100 50, clip]{j8oc16010_wcs/4582-635-Bally_16-images.pdf}
    &\includegraphics[width=0.47\linewidth,  trim=60 50 100 50, clip]{j8oc16010_wcs/w005-514-Bally_16-images.pdf}\\ \hline
    \includegraphics[width=0.47\linewidth,  trim=60 50 100 50, clip]{j8oc17010_wcs/4468-605-Bally_17-images.pdf}
    &\includegraphics[width=0.47\linewidth,  trim=60 50 100 50, clip]{j8oc17010_wcs/LL3-Bally_17-images.pdf}\\ \hline
    \includegraphics[width=0.47\linewidth,  trim=60 50 100 50, clip]{j8oc18010_wcs/LL2-Bally_18-images.pdf}
    &\includegraphics[width=0.47\linewidth,  trim=60 50 100 50, clip]{j8oc24010_wcs/LL4-Bally_24-images.pdf}\\ \hline
   
\end{tabular}
  \caption{Imágenes de \ha{}+\nii{} de los 73 objetos LL detectados en la Nebulosa de Orión, en ellas se puede apreciar la forma de los arcos, la estrella jóven presecuencia principal en el interior, el ajuste de los círculos para los bordes internos y externos de la zona chocada. Imágenes tomadas con la cámara ACS-F658N como parte del programa GO-9825. }
  \label{fig:images} 
\end{figure}

\subsection{Más sobre las observaciones: Pocisiones, distancias, formas y tamaños}
\label{sec:observations}

Como se ha dicho anteriormente usando éstas observaciones  hemos medido parámetros observacionales con el propósito de caracterisar los objetos LL y los choques de proa de los proplyds. En este orden de ideas hemos medido la distancia \(D\) de la fuente a \thC{}, la anchura \(h\) de la cáscara y los radios característicos; \(R_{0}\) y \(R_{c}\) tanto de los límites internos y externos de la cáscara chocada. La tabla~\ref{tab:test} es el resultado final de tales mediciones.\\

\begin{figure}
  \centering
  \includegraphics[width=\linewidth]{luis-programas/will-H-vs-q}
  \caption{Ancho de la cáscara chocada \(h\) dividida entre el radio del choque externo \(r0 = R_{0}\) a lo largo del eje de simetría, para obtnener el ancho relativo \(H\), en función de  \(r0 = R_{0}\) normalizado con \(D\), esto es el término \(q\) que es un indicativo de los tamaños de los objetos LL. El color de los puntos indica la distancia proyectada desde la fuente al Trapecio (ver el panel de colores de la izquierda). Por último, cabe mencionar que el tamaño de los símbolos hace referencia a los valores de los radios de curvatura \(R_{c}\) de los círculos ajustados de los choques externos.}
  \label{fig:thikness}
\end{figure} 

 En la figura~\ref{fig:thikness}, se logra apreciar que los choques de proa situados a grandes distancias  proyectadas desde \thC{}, tienden a tener  pequeños tamaños, puesto que el cociente \(q = R_{0}/D\) es menor para estas distancias\footnote{Se ha tomado \(R_{0}\), como el radio del borde externo de la cáscara chocada.}, esto es en comparación a los objetos más distantes los cuales muestran que el radio relativo \(q\) es más grande, indicando por tanto que son de mayores tamaños. Por otro lado esta figura también nos muestra que los choques de proa más distantes tienden a tener cáscaras más gruesas, es decir que para los objetos de menor tamaño sus cáscaras chocadas son de mayor anchura, esto es debido por un lado a que muchos de estos objetos LL tienen doble cáscara (ver figura~\ref{fig:images}).\\ 

\begin{figure}
  \centering
  \includegraphics[width=\linewidth]{luis-programas/will-A-vs-q}
  \caption{Radios de curvaturas de los choques externos \(R_{c}\) dividido entre \(R_{0}\), a esta fracción la hemos llamado \(A\), en función de \(q\), como una forma para establecer que tan abiertos o cerrados son las alas de los choques LL. La escala de colores representa la distancia de los objetos a \thC{} y el tamaño de los símbolos son un indicativo de la longitud los radios de curvaturas.}
  \label{fig:radii-curvatures}
\end{figure} 

Es de notar que tenemos valores variados para los radios de curvaturas y puesto que con estos parámetros podemos hacernos una idea de la forma de las choques. Entonces tenemos que en la figura~\ref{fig:radii-curvatures} es perceptible que los radios de curvaturas\footnote{También se ha tomado el radio de curvatura externo.} normalizados con los radios axiales \(R_{0}\), esto es la fracción \(A=R_{c}/R_{0}\), aumentan con la distancia a la estrella ionizante y dado que el radio caracterísitico \(R_{0}\) disminuye conforme \(D\) crece como se dijo anteriormente, entonces es coherente argumentar que los choques ubicados en la regiones externas de la nebulosa tienden a mostrar arcos muy abiertos en sus formas, es el caso de los objetos; LL1, LL2, LL3, LL4, LL5, LL6, LL7, w266-558 y w000-400 por citar algunos. Los choques de proa situados en las cercanías de \thC{}, es decir a cortas distancias, vemos que tienen valores de \(A\) más pequeños en comparación a los valores de esta fracción para los objetos más distantes y dado que a estas distancias el radio axial \(R_{0}\) es más grande, entonces nos vamos a enfrentar con el hecho de que los choques de proa ubicados en el interior de la nebulosa, van a tener arcos mas cerrados, como sucede con; 108-326, 142-301, 177-341, 167-317, 168-326N, 168-328, 173-236, 175-321, 180-331, entre otros (ver los valores de los radios de curvatura en la tabla~\ref{tab:test}  y las posiciones de los arcos es las figuras~\ref{fig:position-arc} y~\ref{fig:position-arc-zoom}).\\  

\begin{figure}
  \centering
  \includegraphics[width=\linewidth]{ll-pos-image}
  \caption{Posiciones de los arcos LL superpuestos en una imagen de \ha{} de la Nebulosa de Orión. Los arcos con las flechas representan los objetos LL y los proplyds con sus respectivos choques estacionarios, de nuestro catálogo. Las flechas verdes y violetas indican la orientación de los arcos externos e internos respectivamente. Además se han incluidos los proplyds y otros objetos del catálogo de \citet{Ricci:2008}, donde los puntos de color rojo representan los clásicos proplyds, los puntos de color negro representan los típicos discos de acreción, los de color verde representan jets radiativos sin evidencia de la precencia de discos ionizados y los símbolos de color cian son nebulosas de reflexión sin emisión externa de gas ionizado. El cuadro en la zona del Trapecio de la imagen es la región ampliada en la figura~\ref{fig:position-arc-zoom}.}
  \label{fig:position-arc}
\end{figure}

En la figura~\ref{fig:position-arc} se ilustran las posiciones de los arcos LL de nuestro catálogo en la Nebulosa de Orión, además se muestran las posiciones de los proplyds del catálogo de \citet{Ricci:2008}, incluimos estos objetos para mostrar que muchos proplyds  de la Nebulosa de Orión no tienen choques de proa asociados. Se observa que los objetos LL están repartidos por toda la nebulosa, esto es que exite una gran cantidad de arcos de emisión en el noroeste y suroeste de la misma, una cantidad un poco menor de estos choques se observan en el sureste y es muy poca el número de arcos LL situados en el noreste de la nebulosa; además se indican las orientaciones de los arcos internos y externos de estos objetos. Estas poblaciones; los objetos LL y los choques de proa de los proplyds de la figura~\ref{fig:position-arc} de las que estamos hablando, son aquellos ubicados a largas distancias del Trapecio, puesto que en el interior de la nebulosa el mapa está muy saturado y no se logran ver de manera clara las posiciones y orientaciones de los arcos en esta región. Para eso contamos con la figura~\ref{fig:position-arc-zoom} el cual es una ampliación de la región donde se encuentran las estrellas másivas del Trapecio, en ella es posible ver las posiciones de los proplyds y las orientaciones de sus respectivos choques en el interior de la nebulosa de una forma más clara. Entre otras cosas; las direcciones de las flechas de los arcos radiativos LL en las afueras de la nebulosa sugieren a simple vista que estos arcos están orientados hacia el núcleo de la Nebulosa de Orión, de la misma forma los choques de los proplyds en las regiones internas al parecer están orientados hacia \thC{}.



\begin{figure}
  \centering
  \includegraphics[width=\linewidth]{ll-pos-image-zoom}
  \caption{Posiciones de los arcos LL. Con un zoom de una pequeña área en el núcleo de la nebulosa. Las flechas y los colores de los símbolos representan los mismos conceptos y objetos que en la figura~\ref{fig:position-arc}.}
  \label{fig:position-arc-zoom}
\end{figure}

\newpage
\newcommand\TestTableHeader{
  \hline
Objeto & \(\mathrm{A.R.}\) & \(\mathrm{Decli}\) & \(D\) & \(h\) & \(R_{0}(\mathrm{out})\) & \(R_{0}(\mathrm{in})\) & \(R_{c}(\mathrm{out})\) & \(R_{c}(\mathrm{in})\) \\
  %Object & \(D\) &   \(R_{\mathrm{out}}\) & \(R_{\mathrm{in}}\) \\
  \hline 
}

\begin{longtable}{lcccccccc}
  \caption{Distancias, tamaños y formas de los choques de proa en la Nebulosa de Orión.  \label{tab:test}}\\
  \TestTableHeader\endfirsthead 
  \caption[]{continuación  }\\
  \TestTableHeader\endhead
  \hline \endfoot
  \begin{table}
\begin{tabular}{ccccccccc}
Object & RA & Dec & D & h & R_out & R_in & Rc_out & Rc_in \\
022-635 & 5:35:02.200 & -5:26:35.33 & 286.472 & 0.341 & 1.104 & 0.75 & 4.456 & 2.29 \\
041-637 & 5:35:04.060 & -5:26:37.06 & 267.84 & 0.766 & 1.937 & 1.189 & 4.395 & 3.232 \\
042-628 & 5:35:04.19875 & -5:26:27.59750 & 259.592 & 1.339 & 3.069 & 1.76 & 6.901 & 3.606 \\
049-143 & 5:35:04.945 & -5:21:42.92 & 197.821 & 0.56 & 1.173 & 0.625 & 4.178 & 0.662 \\
051-024 & 5:35:05.131 & -5:20:24.32 & 245.01 & 0.269 & 1.186 & 0.895 & 2.287 & 1.663 \\
065-502 & 5:35:06.53975 & -5:25:01.50750 & 177.288 & 1.214 & 1.422 & 0.491 & 7.737 & 2.31 \\
066-3251 & 5:35:06.56919 & -5:32:51.43000 & 587.491 & 0.0 & 1.068 & -- & 1.588 & -- \\
066-652 & 5:35:06.588 & -5:26:52.38 & 255.852 & 0.102 & 0.028 & -0.075 & 0.575 & 0.6 \\
072-134 & 5:35:07.20375 & -5:21:34.29500 & 174.737 & 2.419 & 4.69 & 2.261 & 17.626 & 7.286 \\
074-229 & 5:35:07.384 & -5:22:28.92 & 144.777 & 0.596 & 1.357 & 0.79 & 1.601 & 0.748 \\
083-435 & 5:35:08.29275 & -5:24:34.85750 & 140.892 & 0.579 & 1.247 & 0.544 & 2.005 & 0.664 \\
101-233 & 5:35:10.133 & -5:22:32.60 & 105.936 & 0.419 & 2.458 & 2.111 & 4.356 & 4.206 \\
102-157 & 5:35:10.25075 & -5:21:57.11750 & 125.297 & 0.45 & 0.796 & 0.402 & 5.03 & 3.542 \\
106-245 & 5:35:10.576 & -5:22:44.69 & 94.703 & 0.38 & 0.627 & 0.228 & 2.476 & 1.121 \\
109-246 & 5:35:10.89575 & -5:22:46.31750 & 89.677 & 0.704 & 1.948 & 1.322 & 10.638 & 7.325 \\
116-3101 & 5:35:11.65419 & -5:31:01.03000 & 463.921 & 0.44 & 1.452 & 1.004 & 2.695 & 1.574 \\
117-421 & 5:35:11.650 & -5:24:21.41 & 92.066 & 0.0 & -- & 0.71 & -- & 0.927 \\
119-3155 & 5:35:11.926 & -5:31:53.30 & 515.101 & 1.002 & 3.015 & 1.951 & 6.727 & 5.11 \\
121-434 & 5:35:12.12175 & -5:24:33.75750 & 95.578 & 0.39 & 0.757 & 0.344 & 1.413 & 0.688 \\
124-131 & 5:35:12.383 & -5:21:31.41 & 126.201 & 1.733 & 4.477 & 2.744 & 5.95 & 10.415 \\
131-046 & 5:35:13.05537 & -5:20:45.78625 & 164.46 & 2.168 & 3.546 & 0.999 & 7.325 & 5.645 \\
132-053 & 5:35:13.202 & -5:20:52.59 & 157.312 & 0.392 & 0.718 & 0.317 & 1.803 & 0.676 \\
136-3057 & 5:35:13.60719 & -5:30:57.56000 & 456.925 & 4.51 & 10.133 & 4.907 & 18.706 & 10.203 \\
138-3024 & 5:35:13.79919 & -5:30:24.40000 & 423.642 & 1.107 & 3.893 & 2.744 & 8.539 & 4.787 \\
142-301 & 5:35:14.158 & -5:23:01.00 & 39.672 & 0.6 & 2.422 & 1.822 & 6.062 & 4.547 \\
154-225 & 5:35:15.367 & -5:22:25.31 & 59.219 & 0.584 & 1.287 & 0.636 & 3.421 & 1.051 \\
154-240 & 5:35:15.383 & -5:22:39.79 & 45.303 & 0.0 & -- & 1.72 & -- & 2.3 \\
158-323 & 5:35:15.831 & -5:23:22.51 & 8.338 & 0.206 & 1.849 & 1.643 & 2.92 & 2.354 \\
159-221 & 5:35:15.934 & -5:22:21.04 & 61.862 & 0.0 & -- & 0.834 & -- & 1.582 \\
160-350 & 5:35:15.958 & -5:23:49.68 & 27.906 & 0.067 & 0.109 & 0.042 & 0.598 & 0.294 \\
161-324 & 5:35:16.056 & -5:23:24.33 & 5.295 & 0.255 & 1.159 & 0.904 & 3.013 & 2.027 \\
162-456 & 5:35:16.182 & -5:24:56.39 & 93.914 & 0.086 & 0.294 & 0.208 & 0.853 & 0.785 \\
163-222 & 5:35:16.303 & -5:22:21.47 & 61.071 & 0.356 & 1.538 & 1.108 & 1.805 & 1.549 \\
163-317 & 5:35:16.282 & -5:23:16.63 & 6.111 & 0.395 & 2.323 & 1.928 & 4.904 & 4.437 \\
165-235 & 5:35:16.475 & -5:22:35.22 & 47.325 & 0.472 & 1.776 & 1.225 & 3.842 & 3.437 \\
166-316 & 5:35:16.607 & -5:23:16.16 & 7.149 & 0.279 & 0.694 & 0.415 & 1.191 & 0.851 \\
167-317 & 5:35:16.739 & -5:23:16.50 & 7.974 & 0.706 & 1.96 & 1.254 & 3.286 & 2.052 \\
168-326 & 5:35:16.839 & -5:23:26.32 & 7.712 & 0.196 & 0.947 & 0.737 & 3.044 & 3.011 \\
168-326N & 5:35:16.835 & -5:23:25.97 & 7.493 & 0.111 & 0.229 & 0.118 & 1.044 & 0.79 \\
168-328 & 5:35:16.757 & -5:23:28.05 & 7.787 & 0.272 & 1.062 & 0.789 & 1.315 & 0.799 \\
169-338 & 5:35:16.880 & -5:23:38.02 & 17.138 & 0.351 & 1.031 & 0.68 & 2.037 & 0.719 \\
170-249 & 5:35:16.967 & -5:22:48.44 & 35.162 & 0.777 & 3.225 & 2.448 & 6.92 & 4.755 \\
173-236 & 5:35:17.352 & -5:22:35.73 & 48.956 & 0.753 & 2.28 & 1.527 & 1.364 & 1.935 \\
173-342 & 5:35:17.324 & -5:23:41.39 & 23.465 & 0.462 & 1.29 & 0.828 & 3.312 & 1.719 \\
175-321 & 5:35:17.458 & -5:23:21.06 & 16.026 & 0.598 & 2.031 & 1.384 & 2.478 & 2.626 \\
177-341 & 5:35:17.667 & -5:23:40.98 & 26.543 & 0.749 & 3.813 & 3.064 & 4.247 & 3.867 \\
178-258 & 5:35:17.819 & -5:22:58.06 & 32.473 & 0.524 & 1.479 & 0.923 & 3.944 & 3.43 \\
180-331 & 5:35:18.033 & -5:23:30.82 & 25.909 & 0.326 & 1.44 & 1.114 & 2.361 & 1.772 \\
189-329 & 5:35:18.868 & -5:23:28.88 & 37.556 & 0.856 & 1.397 & 0.538 & 6.318 & 0.833 \\
203-3039 & 5:35:20.289 & -5:30:39.38 & 440.716 & 3.06 & 5.384 & 1.761 & 17.897 & 14.501 \\
204-330 & 5:35:20.402 & -5:23:30.01 & 60.389 & 0.221 & 0.333 & 0.112 & 1.336 & 1.281 \\
206-043 & 5:35:20.561 & -5:20:43.11 & 171.159 & 0.443 & 1.609 & 1.117 & 2.186 & 2.077 \\
212-400 & 5:35:21.181 & -5:24:00.46 & 80.988 & 0.213 & 1.075 & 0.844 & 0.831 & 0.518 \\
261-3018 & 5:35:26.16875 & -5:30:18.01750 & 440.4 & 2.591 & 4.986 & 2.514 & 33.288 & 3.586 \\
305-811 & 5:35:30.43675 & -5:28:11.23750 & 356.86 & 0.8 & 1.721 & 0.891 & 4.924 & 3.993 \\
308-3036 & 5:35:30.79475 & -5:30:36.25250 & 484.126 & 1.125 & 2.557 & 1.438 & 4.497 & 1.782 \\
344-3020 & 5:35:34.36275 & -5:30:20.56250 & 496.758 & 0.947 & 1.659 & 0.673 & 3.155 & 4.423 \\
362-3137 & 5:35:36.34775 & -5:31:37.75250 & 577.964 & 1.465 & 3.121 & 1.572 & 4.245 & 2.222 \\
4285-458 & 5:34:28.520 & -5:24:57.88 & 721.182 & 0.0 & 1.913 & -- & 4.344 & -- \\
4468-605 & 5:34:46.75775 & -5:26:04.81750 & 471.3 & 1.199 & 2.469 & 1.331 & 7.109 & 2.223 \\
4578-251 & 5:34:57.79275 & -5:22:51.09500 & 279.493 & 0.533 & 1.846 & 1.188 & 3.52 & 2.074 \\
4582-635 & 5:34:58.16675 & -5:26:35.12750 & 333.372 & 0.455 & 1.106 & 0.682 & 2.931 & 2.056 \\
LL1 & 5:35:05.63675 & -5:25:19.44750 & 198.626 & 1.128 & 3.057 & 1.904 & 8.721 & 7.132 \\
LL2 & 5:34:40.860 & -5:22:42.20 & 532.124 & 1.133 & 4.035 & 2.074 & 27.93 & 13.844 \\
LL3 & 5:34:40.807 & -5:26:38.54 & 566.331 & 1.83 & 3.119 & 1.284 & 6.544 & 3.085 \\
LL4 & 5:34:42.719 & -5:28:37.20 & 593.107 & 0.992 & 2.415 & 1.422 & 11.787 & 4.952 \\
LL5 & 5:35:31.39775 & -5:28:16.35750 & 369.536 & 1.487 & 2.963 & 1.456 & 11.577 & 3.517 \\
LL6 & 5:35:32.86575 & -5:30:21.45250 & 485.816 & 2.049 & 3.628 & 1.626 & 29.899 & 14.201 \\
LL7 & 5:35:35.126 & -5:33:49.16 & 686.232 & 1.46 & 6.998 & 5.533 & 17.958 & 10.306 \\
w000-400 & 5:34:59.56575 & -5:24:00.14500 & 254.035 & 0.675 & 1.465 & 0.795 & 4.369 & 2.432 \\
w005-514 & 5:35:00.471 & -5:25:14.21 & 262.637 & 0.424 & 1.672 & 1.171 & 2.239 & 2.213 \\
w005-514 & 5:35:00.46775 & -5:25:14.29750 & 262.718 & 0.437 & 1.651 & 1.201 & 3.449 & 2.032 \\
w012-407 & 5:35:01.17375 & -5:24:06.67750 & 231.47 & 1.234 & 2.289 & 0.953 & 6.743 & 4.264 \\
w014-414 & 5:35:01.37175 & -5:24:13.36750 & 229.954 & 0.696 & 1.211 & 0.373 & 2.185 & 1.615 \\
w030-524 & 5:35:03.00375 & -5:25:24.35750 & 234.087 & 0.315 & 0.626 & 0.288 & 2.559 & 1.541 \\
w044-527 & 5:35:04.427 & -5:25:27.39 & 217.945 & 0.851 & 2.13 & 0.78 & 3.25 & 1.527 \\
w069-601 & 5:35:06.90775 & -5:26:00.57750 & 212.194 & 0.423 & 0.853 & 0.405 & 2.723 & 1.773 \\
w073-227 & 5:35:07.26975 & -5:22:26.49750 & 147.268 & 0.696 & 1.626 & 0.811 & 6.462 & 3.979 \\
w266-558 & 5:35:26.618 & -5:25:58.29 & 218.155 & 0.775 & 1.88 & 1.127 & 7.761 & 1.506 \\
\end{tabular}
\end{table}

\end{longtable}

\section{Resultados Astrofísicos}
\label{sec:results}


Las observaciones de las que hemos hablado hasta el momento, junto con los parámetros observacionales medidos nos han permitido esbosar algunos parámetros astrofísicos de los arcos de emisión, con el objetivo de intentar comprender los fenómenos físicos que en nuestros objetos están ocurriendo. En este orden de ideas hemos determinado la densidad en la cáscara chocada, la presión de estancamiento y el flujo de momento, que es lo que veremos a continuación.

\subsection{Densidad}
\label{sec:density}

\begin{figure}
  \centering
   \includegraphics[width=\linewidth, clip]{luis-programas/will-nshell-vs-D.pdf}
  \caption{Densidad electrónica corregida por extinción en función de la distancia proyectada a \(\theta^1\ \text{Ori}\ \text{C}\), obtenida a partir de las observaciones de Bally, es decir usando el  brillo superficial de \(\ha\) en la cáscara chocada para determinarla. El tamaño de los puntos indica que tan grande es el camino de visión en la zona chocada si los comparamos entre si. Las flechas en la parte superior izquierda representan la distancia corregida de los arcos de emisión a \thC{}, para los ángulos de inclinación; \(i = 30^{\circ}, 45^{\circ}, 60^{\circ}, 75^{\circ}\).  Por otro lado la escala de colores representa el brillo superficial de los objetos en unidades de [\(\mathrm{erg~s^{-1}~cm^{-2}~sr^{-1}}\)]. }
  \label{fig:density}
\end{figure}

Dado que a partir de las observaciones hemos podido determinar el brillo superficial de \ha{} realizando un poco de fotometría como se puede apreciar previamente y además como tenemos el camino de la línea de visión en la cáscara chocada (\(\Delta\zeta\)), dada por la Ec. \ref{eq:vision},  hemos determinado la densidad de núcleos de Hidrógenos (\(n\))  usando la ecuación \ref{eq:densidad}, específicamente en esta región (en la cáscara) para los diferentes objetos LL y los choques de los proplyds. Hay que subrayar que hemos usado cómo distancia a la Nebulosa de Orión 436 Pc para hacer las respectivas conversiones de unidades.\\ 

En la figura \ref{fig:density} se logra apreciar que los objetos que están más cercas de las estrellas masivas del Trapecio, presentan una densidad electrónica mayor que los que se encuentran a las afueras de la nebulosa, entonces estas altas densidades electrónicas provocan que el brillo superficial, generado a partir de la emisión de la líneas de recombinación de \ha{} sea mayor en las proximidades de \thC{}. Y es que en esta región domina la emisión de \ha, es decir hay poca contaminación por las líneas de  \nii{}.\\


\subsection{Presión hidrodinámica y presión térmica }
\label{sec:pressure}

\begin{figure}
  \centering
  \includegraphics[width=\linewidth, clip]{luis-programas/will-Pshell-vs-D.pdf}
  \caption{Los símbolos indican la presión térmica en la cáscara chocada  y las líneas representan la presión hidrodinámica generada por el viento estelar, ambas en unidades de [\(\mathrm{dynas~cm^{-2}}\)], en función de la distancia proyectada a \thC{}. La línea continua representa la presión ram del viento estelar en función de la distancia suponiendo que no hay inclinación, esto es para \(i = 0^{\circ}\), la línea discontinua representa la misma presión ram para una distancia proyectada cuando se cambia el ángulo de inclinación desde la distancia (\(i = 0^{\circ}\)), es decir para un ángulo de inclinación de \(i = 30^{\circ}\) y la línea de puntos también representa la presión ram para una distancia proyectada con un ángulo de inclinación de \(i = 60^{\circ}\). La escala de colores indica el brillo superficial de \ha{}. }
 \label{fig:pressure}
\end{figure}

Usando la ecuación \ref{eq:presion-cascara} de \S\ref{sec:pressur-thermal} se estimó la presión térmica de la cáscara chocada, a partir de la densidad electrónica estimada en \S\ref{sec:density} y suponiendo una temperatura para esa región de \(10^{4}~\K\).\\

 Por otro lado usando la Ec. \ref{eq:presion-externa} de la misma sección determinamos la presión hidrodinámica ejercida para cada objeto por el viento estelar hipersónico de la estrella másiva \thC{} del Trapecio (usando la distancia \(D\) del objeto en cuestión a \thC{}), para una cierta tasa de pérdida de masa de \(\dot{M} = 3.5 \times 10^{-7}~\msolagno\) y una velocidad de \(v = 1200~\mathrm{km~s^{-1}}\). \\

Ahora bien, la figura \ref{fig:pressure} es el resultado de tales estimaciones. En ella estamos comparando las presiones en las cáscara chocada de los objetos LL y de los proplyds (símbolos de colores en la gráfica), con la presión ram generada por el viento estelar (lineas continuas  y discontinuas  de color  negro en la gráfica). Se observa  que la presión térmica es mayor en los objetos, que están dentro de la nebulosa,  es decir en los proplyds conocidos, a su vez esta  coincide con la presión ram del flujo de la estrella masiva, indicando que los choques de los proplyds en el interior están confinados por el hipersónico viento estelar, es posible argumentar esto considerando el equilibro de presiones (ver Ec. \ref{eq:igualda-presion}). Lo contrario sucede con los arcos hiperbólicos en las afueras de la nebulosa, puesto que la presión en la zona chocada no coincide con la presión del viento estelar, por tanto esto nos lleva a pensar que estos objetos no están interactuando con el viento estelar, sino más bien con el transónico flujo de champaña fotoevaorado proveniente del núcleo de la nebulosa.  

\subsection{Flujo de momento interno: \(\dot{M}_{w}\text{V}_{w}\) }
\label{sec:momentum}

\begin{figure}
  \centering
  \includegraphics[width=\linewidth, clip]{luis-programas/will-MdotV-vs-D.pdf}
  \caption{Flujo de momento interno  en función de la distancia proyectada. El color de los símbolos indican: Rojo; son verdaderos proplyds, naranja; podrían tratarse de proplyds, negro; no son proplyds. El tomaño de los simbolos representan el tamaño de la línea de visión (\(\Delta\zeta\)) en la cáscara chocada.  }
 \label{fig:flow}
\end{figure}


Hemos determinado el flujo de momento interno para los objetos de nuestro catálogo a partir de las presiones de estancamiento, presiones con las cuales se obtuvo la ecuación \ref{eq:momentum}, con esta ecuación fue posible determinar el ya mencionado flujo de momento, en este orden de ideas se utilizaron los valores de la presión térmica determinados arriba, junto con los valores de los radios \(R_{0}\) internos de los choques LL para este fin. Asi que en la figura \ref{fig:flow} se ilustran dichos resultados.\\

No obstante, en la figura~\ref{fig:flow} se logra apreciar que para los objetos clasificados como proplyds en nuestro catálogo (símbolos de color rojo) y que están a cortas distancias del Trapecio presentan grandes pérdida de masa, es decir  muestran fuertes vientos, esto es debido probablemente a un incremento en el tamaño de sus discos. Ahora bien, en los proplyds y supuestos proplyds (símbolos de color naranja) ubicados en las regiones externas de la nebulosa, se observa una disminución del parámetro \(\dot{M}_{w}V_{w}\). Esto es debido a que el flujo de fotones FUV (ultravioletas lejano), responsables directo de la destrucción inevitable de los discos de acreción, decrecen con la distancia a la estrella masiva \thC{}. Por último, para el caso caso de los objetos que no son proplyds (símblos de color negro) situados en las regiones externas de la nebulosa, se observa que tienden a tener fuertes vientos. Aún no sabenos a que se debe este fenómeno.\\ 

\section{Discusión}
\label{sec:disc}

\begin{figure}
  \centering
  \includegraphics[width=\linewidth, clip]{luis-programas/will-PA-vs-PA}
  \caption{Angulo entre el eje del choque de proa y la direción a radial, este último es en dirección a \thC{}, en función de la posición angular de \thC{} a la fuente en el plano del cielo, es decir tomando las coordenads cartesianas (x, y). La línea azul y vertical representa el ángulo ``0'' entre las dos posiciones angulares. La escala de colores representa la distancia proyectada desde los objetos LL al Trapecio y el tomaño de los símbolos indica el tamaño del radio de curvatura externo.}
 \label{fig:pos-angular}
\end{figure}

Las mediciones de las posiciones de los 73 arcos de emisión detectados en la Nebulosa de Orión han permitido indicar las orientaciones de los choques, que siguen la dirección del flujo externo de la nebulosa, como resultado intrínsico de la interacción de este con el viento interno de la estrella T-Tauri o proplyd. Deacauerdo a las orientaciones mostradas en las  figuras \ref{fig:position-arc} y \ref{fig:position-arc-zoom} de los arcos en la nebulosa, podemos pensar que el flujo proveniente del núcleo de la nebulosa es aproximadamente radial, como se ha interpretado en la sección \S\ref{sec:images}.\\

 No obstante, en la figura \ref{fig:pos-angular} se ilustran de manera más estricta, que tanto siguen la dirección radial las orientaciones de los arcos radiativos en las regiones internas y externas  de la nebulosa, considerando la dirección radial; como la línea imaginaria que va desde la fuente hasta \thC{}. Para ello, con los valores medidos de las posiciónes angulares de las fuentes a \thC{}, es decir el ángulo formado por la dirección radial en el plano del cielo (x, y), junto con los valores medidos de las posiciónes angulares de los ejes de los choques o eje de simetría en el plano del cielo, se ha estimado el desplazamiento angular del eje del choque con respecto a la línea imaginaria trazada desde la fuente a \thC{}, tomando este útlimo como eje de referencia\footnote{A excepción de los choques de proa producidos por  la interacción de los vientos de dos proplyds, donde se ha tomado como eje de referencia, la dirección de un proplyd con respecto al otro.}, en otras palabras se ha calculado el ángulo entre el eje del choque y la dirección radial. Los resultados han mostrado que las direcciones de los ejes de los choques no son estrictamente radiales, puesto que están desplazado en un intervalo  que va desde \(0^{\circ}\) a \(-50^{\circ}\) en el noroeste y noreste de la nebulosa, mientras que los ejes de los arcos ubicados en el sureste y suroeste está despazados en un intérvalo de ángulo que va desde \(0^{\circ}\) a \(50^{\circ}\). Estos ángulos positivos y negativos en los diferentes cuadrantes de la nebulosa indican un desplazamiento hacia la izquierda de la dirección radial, de los ejes de los choques de proa. \\

\begin{figure}
\centering
\begin{tabular}{l l}
  (\textit{a})\\
  \includegraphics[width=0.76\linewidth]{proplyd-star-ratio}\\
  (\textit{b})\\
  \includegraphics[width=0.76\linewidth]{bowshock-proplyd-ratio}
  \end{tabular}  
  \caption{(\textit{a})~Fracción de todas las estrellas opticamente visibles que son proplyds (simbolos circulares oscuros) o que tienen choques de proa (símbolos cuadrados claros) como una función de la separación proyectada desde el Trapecio. (\textit{b})~Relación entre el número de choques de proa y el  número de proplyds en función de la separación proyectada desde el Trapecio. Los símbolos circulares y oscuros representan todos los choques de nuestro catálogo (a excepción de los choques producidos pos la interacción de dos proplyds) mientras que los símbolos cuadrados y claros indican estos mismos choques de proa pero sólo aquellos asociados con  conocidos y supuestos proplyds.}
\label{fig:bow-proplyd-star-ratios}
\end{figure}

En la figura~\ref{fig:bow-proplyd-star-ratios}, se observa que la fracción de proplyds entre el número de estrellas caen relativamente con facilidad con la distancia proyectada a \thC{}. Mostrando una repentina caida después de unos 200~arcsec.\\

Por otro lado la fracción entre los choques de proa y los proplyds parecen tener tres picos separados. En este orden de ideas, para distancias muy pequeñas se puede ver que el primer pico corresponde a la interacción de los vientos estelar con el viento ionizado de los proplyds, a continuación hay una escaces de choques de proa hasta el segundo pico cerca de unos 4~arcmin. Por último se puede apreciar que  a grandes distancias, donde podría haber un tercer pico dominan el número de objetos que no son proplyds.\\

No obstante una explicación alternativa para lo planteado anteriormente, podría ser que en este tercer pico todos los objetos son proplyds, pero como se piensa que a grandes distancias no se sabe con certeza que clase de objetos están originando los vientos en una escala más pequeña, entonces estaríamos subestimando  la fracción de proplyds para grandes estas distancias.\\

También hay evidencia de tres distintas poblaciones de la distribución azimutal alrrededor del Trapecio. Para el grupo ubicado a 4~arcmin de separación son principalmente aquellos objetos del oeste de la nebulosa, mientras que los objetos más distantes son principalmente los del sur.    

%\bibliography{luis-ref}

%\end{document}
