%\documentclass{article}
%\usepackage[utf8]{inputenc}
%\usepackage{amsmath}
%\usepackage{natbib}
%\usepackage{graphicx}
%\usepackage{astrojournals} % Necesario para nombres de revistas en luis-ref.bib
%\usepackage[spanish, es-minimal]{babel}
%\usepackage{longtable}
%\usepackage{geometry}
%\usepackage{multirow, array}
%\newlength\figwidth
%\bibliographystyle{apj}

%\title{Catalog of stationary bowshock arcs in the Orion Nebula}

%\author{
  %Alumno: Luis Angel Gutiérrez Soto\\
  %Tutor: Dr. William Henney
%}
%\begin{document}
%\maketitle

%\chapter{Conclusiones}
\label{chap:conclu}

En la parte observacional, hemos realizado un catálogo muy completo  con un total de 73 objetos (ver figura \ref{fig:bow-proplyd-star-ratios}), los cuales hemos clasificado como arcos de emisión estacionarios (objetos LL y choques de proa asociados a proplyds), detectados en unas imágenes en el óptico del \textit{HST} de la Nebulosa de Orión. De esos 73 objetos 20 no han sido reportados previamente en la literatura. Encontramos que los choques de proa están distribuidos en casi toda la nebulosa, esto es que hay cierta cantidad de  estos choques de proa asociados a proplyds que habitan en las regiones internas de la nebulosa, un grupo de choques de proa se situan en el oeste de la nebulosa a más menos 640~arcsec de separación del Trapecio y un último  grupo de los cuales todos son objetos LL, es decir arcos de emisión sin supuetos proplyds están situados en el sur de la nebulosa a grandes distancias de \thC{}.\\      

Por otro lado, ajustando círulos en los límites externos e internos de la cáscara chocada, hemos podido clasificar la forma de los choques de proa en dos grupos, puesto que hemos encontrado diferencias muy significativas entre los  objetos ubicados cerca de la estrella ionizadora  y los que se encuentra a más grandes distancias. Entonces un primer grupo cercano, el cual corresponde a  la interacción del viento estelar  hipersónico con el viento de gas ionizado de los proplyds (todos los objetos en el interior de la nebulosa son proplyds), muestran que sus choques tienen forma relativamente cerrada, esto probablente se deba a que los vientos de los propyds no son isotrópicos, mientras un segundo grupo en el que sus miembros se situan en regiones más lejanas del Trapecio. Sus arcos muestran formas más abiertas e hiperbólicas, dado que a grandes distancias de \thC{}, es decir en las afueras de la nebulosa, la población de estrellas que domina son las estrellas T-Tauri con sus respectivos discos de acreción, los cuales producen un viento isotrópico. Entonces esto podría explicar en parte la forma abierta de sus arcos.\\

Usando  los ajustes hecho a las formas de los arcos en las imágenes, medimos el brillo superficial de \ha{} en la cáscara chocada corregido por el fondo, encontrando que en el interior de la nebulosa las líneas de \nii{} son casi inexistentes. Por otro lado basandonos en las orientaciones de los choques de proa hemos llegado a la conclusión, de que el flujo proveniente del núcleo de la nebulosa es aproximadamente radial, indicando por tanto que el flujo en estas regiones no es turbulento ni desordenado. \\       

De los resultados astrofísicos, hemos aprendido que los choques de proa de los proplyds conocidos, en el interior de la Nebulosa de Orión están confinados por el viento de la estrella masiva \thC{}, puesto que esta estrella es la que domina la emisión en esa región de la nebulosa, mientras que en las afueras de la nebulosa hemos establecido que el flujo de las estrella T-Tauri o de los proplyds, interacciona con el transónico flujo de champaña de gas ionizado, provocando  que en estas regiones el borde  externo de la cáscara chocada sea más brillante, explicando el por qué, los arcos son más abiertos en esta región. No obstante hemos llegado a estas conclusiones utilizando la idea de que los choque de proa son estacionarios, esto es considerando un equilibrio entre las presiones hidrodinámicas de los flujos y la presión térmica de la cáscara chocada. \\

Con las presiones de estancamiendo fue posible determinar el flujo de momento de los vientos internos (esto es de los proplyds u objetos LL dependiendo de cual sea el caso), en el que hemos encontrado que los objetos en el interior de la Nebulosa de Orión, es decir de proplyds conocidos como se ha dicho anteriormente presentan grandes pérdida de masa, una explicación para esto sería que los discos en las cercanías de la estrella ionizadora tienen grandes tamaños. Ahora,  se observa que la tasa de pérdida de masa decae secuencialmente con la distancia, para aquellos objetos  en las regiones externas de la nebulosa, identificados como proplyds y supuestos proplyds. Finalmente las fuentes en las regiones más distantes del Trapecio, los cuales hemos clasificado como no proplyds en las afueras de la nebulosa, presenten fuertes vientos estelares, lo cual implica que el tamaño de sus discos están jugando un papel importante.\\

Por último es importante mencionar deacuerdo a todos los resultados obtenidos y las conclusiones planteadas anteriormente, que nuestros arcos de emisión están dividos en dos grandes grupos. Un grupo cercano en el interior de la de la nebulosa en el que la fuente son proplyds ya conocidos, en el que el flujo interno tiene su origen en el frente de  ionización y un grupo lejano formado por proplyds confirmados y por muchos objetos que no lo son. En este los más grandes y brillantes arcos tienden a estar asociados con estrellas jóvenes particularmente luminosas, lo que nos lleva pensar que los vientos intrínsicos de los discos de acreción de las estrellas T-Tauri son muy importantes como mecanismo que moldean  choques de proa por la  interacción con el flujo de champaña fotoevaporado del núcleo de la nebulosa.  

 
%\end{document}